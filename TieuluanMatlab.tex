%Phần thiết đặt trang
\documentclass[12pt,a4paper]{article}
\usepackage[utf8]{vietnam}
\usepackage{amsfonts}
\usepackage{amsmath}
\usepackage{amssymb}
\usepackage[left=2cm,right=2cm,top=2cm,bottom=2cm]{geometry}
\setlength{\parindent}{0pt}
\usepackage{parskip}
\setlength{\parskip}{0.5em}

%Phần thiết kế khung code nhập liệu
\usepackage{listings}
\usepackage{color}

\definecolor{dkgreen}{rgb}{0,0.6,0}
\definecolor{gray}{rgb}{0.5,0.5,0.5}
\definecolor{mauve}{rgb}{0.58,0,0.82}

\lstset{frame=tb,
  language=C++,
  aboveskip=3mm,
  belowskip=3mm,
  showstringspaces=false,
  columns=flexible,
  basicstyle={\small\ttfamily},
  numbers=left,
  numberstyle=\small\color{gray},
  keywordstyle=\color{blue},
  commentstyle=\color{dkgreen},
  stringstyle=\color{mauve},
  breaklines=true,
  breakatwhitespace=true,
  tabsize=3
}

%Nội dung chính
\begin{document}
	
	
	%Bìa để sau
	
	%Mục lục
	Tiểu luận môn học Phần mềm Toán học

“Một số kiến thức về phần mềm MATLAB”

Chương 1: Cơ bản về MATLAB

1.1  Không gian làm việc

1.2  Biến, câu giải thích, chấm câu

1.3  Các hằng số, các phép toán cơ bản, các hàm toán học.

1.4  Quản lý tập, M-file, M-hàm

(C1 $\rightarrow$ C4)

Chương 2: Các thao tác và phép toán với mảng

(C6 và C7)

Chương 3: Vòng lặp điều khiển (C11)

Chương 4: Đồ thị trong mặt phẳng và trong không gian

(Bổ sung nhiều ví dụ và bài tập)
	%Test cho code
	\begin{lstlisting}
	// Hello.java
	for (i:1);
	import javax.swing.JApplet;
	import java.awt.Graphics;

	public class Hello extends JApplet {
    	public void paintComponent(Graphics g) {
        	g.drawString("Hello, world!", 65, 95);
    	}    
	}
	\end{lstlisting}
	
\end{document}